\documentclass[tikz,border=10pt]{standalone}
\usepackage{tikz}
\usetikzlibrary{positioning}
\usetikzlibrary{shapes.geometric}% tikz node 形状的库
\usetikzlibrary{patterns,quotes,arrows.meta,graphs,shapes.misc}
\usetikzlibrary{graphdrawing}
\usegdlibrary{layered} 
\usepackage{tikz-feynman}
\begin{document}

\tikz [>={Stealth[round]}, black!50, text=black, thick,
every new ->/.style = {shorten >=1pt},
graphs/every graph/.style = {edges=rounded corners},
skip loop/.style = {to path={-- ++(0,#1) -| (\tikztotarget)}},
hv path/.style = {to path={-| (\tikztotarget)}},
vh path/.style = {to path={|- (\tikztotarget)}},
nonterminal/.style = {
rectangle, minimum size=6mm, very thick, draw=red!50!black!50, top color=white,
bottom color=red!50!black!20, font=\itshape, text height=1.5ex,text depth=.25ex},
terminal/.style = {
rounded rectangle, minimum size=6mm, very thick, draw=black!50, top color=white,
bottom color=black!20, font=\ttfamily, text height=1.5ex, text depth=.25ex},
shape = coordinate
]
\graph [grow right sep, branch down=7mm, simple] { %simple 的作用是,让一对node之间只能有一条连线,后发优先.
/ -> unsigned integer[nonterminal] -- p1 -> "." [terminal] -- p2 -> digit[terminal] --
p3 -- p4 -- p5 -> E[terminal] -- q1 ->[vh path]
{[nodes={yshift=7mm}]
"+"[terminal], q2, "-"[terminal]
} -> [hv path]
q3 -- /unsigned integer [nonterminal] -- p6 -> /;
p1 ->[skip loop=5mm] p4;
p3 ->[skip loop=-5mm] p2;
p5 ->[skip loop=-11mm] p6;
q1 -- q2 -- q3; % 迫使这些 edge 平滑
};

\end{document}


\graph [layered layout, grow=right, sibling distance=5mm, simple]