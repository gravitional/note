\documentclass[tikz,border=10pt]{standalone}
\usepackage{tikz}
\usetikzlibrary{positioning}
\usepackage{tikz-feynman}
\begin{document}

\begin{tikzpicture}[
		decuplet/.style={ % 自定义一个重子的双线
				double distance=1pt,
				postaction={decorate}, decoration={
						markings, mark=at position .6 with {
								\arrow{Triangle[angle=40:2pt 3]}
							},
					}
			}
	]
	\begin{feynman}
		%% fig d
		\vertex (d1) at (0,0);
		\vertex[right =1cm  of d1] (d2);
		\vertex[right =2cm  of d1] (d3);
		\vertex[right =3cm  of d1, dot,anchor=center] (d4){};
		\vertex[right =4cm  of d1] (d5);
		\node[above =0.5 cm  of d5] {$k$};
		%% fig e
		\vertex[above right =0 cm and 5 cm of d1] (e1);
		\vertex[right =1cm  of e1, dot,anchor=center] (e2){};
		\vertex[right =2cm  of e1] (e3);
		\vertex[right =3cm  of e1] (e4);
		\vertex[right =4cm  of e1] (e5);
		\node[above =0.5cm of e5] {$k$};
		% 对各个顶点连线
		\diagram*{
		{
				[edge=plain]
				(d1) --  (d2), (d4)-- (d5),
				(e1) --  (e2), (e4)-- (e5),
			},
		% 介子连线
		{
		[edge= charged scalar]
        (d2) --[half right](d4),
        (e2) --[half right](e4),
		}
		};
			%% 添加重子双线
            \draw[decuplet]	 (d2) -- (d4);
            \draw[decuplet]	 (e2) -- (e4);
	\end{feynman}
\end{tikzpicture}

\end{document}

