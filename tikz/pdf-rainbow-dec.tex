\documentclass[tikz,border=10pt]{standalone}
\usepackage{tikz}
\usetikzlibrary{positioning}
\usepackage[compat=1.1.0]{tikz-feynman}  % texdoc tikz-feynman 
\tikzfeynmanset{/tikzfeynman/warn luatex=false}

\input tikz-auto-mark-nodes
\tikzset{
    every auto mark/.append style={
            pin position=20
        },
    decuplet/.style={ % 自定义一个重子的双线
            double distance=1pt,
            postaction={decorate}, decoration={
                    markings, mark=at position .6 with {
                            \arrow{Triangle[angle=40:1pt 3]}
                        },
                }
        }
}

\begin{document}

\begin{tikzpicture}[auto mark]
    \begin{feynman}
        %% fig i
        \vertex (i1) at (0,0);
        \vertex[right =1cm  of i1] (i2);
        \vertex[right =2cm  of i1,crossed dot,anchor=center] (i3){};
        \vertex[right =3cm  of i1] (i4);
        \vertex[right =4cm  of i1] (i5);
        \node[above =0.5 of i5] {$i$};
        % 对各个顶点连线
        \diagram*{
        {
        [edge=plain]
        (i2) --[
        momentum={
                \scriptsize \(p-k\)
            }
        ]  (i3) --[
        momentum={\scriptsize \(p^{\prime}-k\)}
        ] (i4),
        },
        {
                [edge=plain]
                (i1) --  (i2),(i4)-- (i5),
            },
        % 介子连线
        {
        [edge= charged scalar]
        (i2) --[
        half right, momentum'={\scriptsize \(k\)}
        ](i4),
        }
        };
        %% 添加重子双线
        \draw[decuplet]	 (i2) -- (i3);
        \draw[decuplet]	 (i3) -- (i4);
    \end{feynman}
\end{tikzpicture}

\end{document}

