%%!TEX root = ..\..\main.tex
%%!TEX encoding = UTF-8 Unicode

% 翻译:张建东
% 初校:

%%++++++++++++++++++++
\documentclass{ctexart} 
\usepackage{axodraw2}
\usepackage{amsfonts}
\usepackage{amsmath}
\usepackage{latexsym}
\usepackage{amssymb}
\usepackage{makeidx}
\usepackage{amsmath}
\usepackage{graphicx}
\usepackage{graphicx,graphics,color}
\usepackage{multirow}
\usepackage{axodraw2}
\usepackage{slashed}
%%++++++++++++++++++++

\begin{document}


\part{附录C:费曼规则}

这里我们收集了各章中的费曼规则。

绘制所有可能的图。
为每条线标记上动量。
如果合适,还要为每条描述矢量场的线标上一个入射和出射的洛伦兹指标,
为每条描述在内部对称性下变换的场的线标上入射或出射的内部指标,
以及诸如此类的事情。
在每个顶点处均有动量守恒。
内线的动量将与用测度$\int[d^4p/(2\pi)^4]$积分。
对于每个闭合的费米子圈都有一个相应的$(-1)$因子。
外线将被截肢。
对于入射的费米子线,写下$u(p,s)$,
而对出射的费米子线,则写下$\bar{u}(p^\prime,s^\prime)$。
对于入射的反费米子,写下$\bar{v}(p,s)$,
而对出射的费米子线,则写下$v(p^\prime,s^\prime)$。
如果存在使图保持不变的对称变换,那我们就必须担心臭名昭著的对称因子。
由于我不信任各种教科书中的汇编,
因此我会从头算出对称因子,而这就是我建议你去做的。

\section{标量场与狄拉克场相互作用}

\begin{equation}\label{equ:app.c.1}
	\mathfrak{L}=\bar{\psi}(i\gamma^\mu\partial_\mu-m)\psi
	+\frac{1}{2}[(\partial\varphi)^2-\mu^2\varphi^2]
	-\frac{\lambda}{4!}\varphi^4+f\varphi\bar{\psi}\psi
\end{equation}

标量传播子:\\
\begin{align}\label{equ:app.c.2}
	\begin{minipage}[c][5ex][c]{8em}
		\begin{axopicture}(100,15)
			\Line[arrow,dash](0,5)(100,5)
			\Text(50,15){$k$}
		\end{axopicture}	
	\end{minipage}
	&\quad&
	\frac{i}{k^2-\mu^2+i\varepsilon}
			%%%
	\end{align}

标量顶点:\\
\begin{align}\label{equ:app.c.3}
\begin{minipage}[c][16ex][c]{8em}
	\begin{axopicture}(80,80)
		\Black{\Line[dash](0,0)(80,80)}
		\GCirc(40,40){2}{0}
		\Black{\Line[dash](80,0)(0,80)}
	\end{axopicture}	
\end{minipage}
&\quad&
		-i\lambda
		%%%
\end{align}

费米子传播子:\\
\begin{align}\label{equ:app.c.4}
	\begin{minipage}[c][5ex][c]{8em}
		\begin{axopicture}(100,20)
			\Line[arrow,dash](0,10)(100,10)
			\Text(50,20){$p$}
		\end{axopicture}	
	\end{minipage}
	&\quad&
	\frac{i}{\slash{p}-m+i\varepsilon}=
		i\frac{\slash{p}+m}{\slash{p}^2-m^2+i\varepsilon}
	\end{align}

标量-费米子顶点:\\
\begin{align}\label{equ:app.c.5}
	\begin{minipage}[c][8ex][c]{8em}
		\begin{axopicture}(150,60)
			\Line[arrow](10,10)(80,10)
			\Line[arrow](80,10)(150,10)
			\Line[dash](80,10)(80,60)
			\end{axopicture}	
	\end{minipage}
	&\quad&
	if
	\end{align}

初始外线费米子:
\begin{equation}\label{equ:app.c.6}
	u(p,s)
\end{equation}

终止外线费米子:
\begin{equation}\label{equ:app.c.7}
	\bar{u}(p,s)
\end{equation}

初始外线反费米子:
\begin{equation}\label{equ:app.c.8}
	\bar{v}(p,s)
\end{equation}

终止外线反费米子:
\begin{equation}\label{equ:app.c.9}
	v(p,s)
\end{equation}

\section{矢量场与狄拉克场相互作用}

\begin{equation}\label{equ:app.c.10}
	\mathfrak{L}=\bar{\psi}(i\gamma^\mu(\partial_\mu-ieA_\mu)-m)\psi
	-\frac{1}{4}F_{\mu\nu}F^{\mu\nu}-\frac{1}{2}\mu^2A_\mu A^\mu
\end{equation}

矢量玻色子传播子:\\
\begin{align}\label{equ:app.c.11}
	\begin{minipage}[c][5ex][c]{8em}
		\begin{axopicture}(110,32)
			\Photon(10,20)(100,20){2}{10}
			\Text(55,32){$k$}
			\Line[arrow,arrowscale=1.4](54,20)(56,20)
			\end{axopicture}
	\end{minipage}
	&\quad&
	\frac{i}{k^2-\mu^2}\left(\frac{k_\mu k_\nu}{\mu^2}-g_{\mu\nu}\right)
	\end{align}

光子传播子(其中$\xi$是任意规范参数):\\
\begin{align}\label{equ:app.c.12}
	\begin{minipage}[c][5ex][c]{8em}
		\begin{axopicture}(100,32)
			\Photon(10,20)(100,20){2}{10}
			\Text(55,32){$k$}
			\Line[arrow,arrowscale=1.4](54,20)(56,20)
			\end{axopicture}
	\end{minipage}
	&\quad&
	\frac{i}{k^2}\left[(1-\xi)\frac{k_\mu k_\nu}{\mu^2}-g_{\mu\nu}\right]
	\end{align}

矢量玻色子-费米子顶点:\\
\begin{align}\label{equ:app.c.13}
	\begin{minipage}[c][8ex][c]{8em}
		\begin{axopicture}(150,60)
			\Line[arrow](10,10)(80,10)
			\Line[arrow](80,10)(150,10)
			\Photon(80,10)(80,60){2}{10}
			\Text(88,60){$\mu$}
			\end{axopicture}
	\end{minipage}
	&\quad&
	ie\gamma^\mu
	\end{align}

初始外线矢量玻色子:
\begin{equation}\label{equ:app.c.14}
	\varepsilon_\mu(k)
\end{equation}

终止外线矢量玻色子:
\begin{equation}\label{equ:app.c.15}
	\varepsilon_\mu(k)^\ast
\end{equation}

\section{非阿贝尔规范理论}

规范玻色子传播子:\\
\begin{align}\label{equ:app.c.16}
	\begin{minipage}[c][5ex][c]{8em}
		\begin{axopicture}(100,32)
			\Photon(10,20)(100,20){2}{10}
			\Text(55,32){$k$}
			\Line[arrow,arrowscale=1.2](54,20)(56,20)
			\end{axopicture}
	\end{minipage}
	&\quad&
	\frac{i}{k^2}\left[(1-\xi)\frac{k_\mu k_\nu}
		{\mu^2}-g_{\mu\nu}\right]\delta_{ab}
	\end{align}

鬼场传播子:\\
\begin{align}\label{equ:app.c.17}
	\begin{minipage}[c][5ex][c]{8em}
		\begin{axopicture}(100,18)
			\Line[arrow,dash](0,10)(100,10)
			\Text(50,18){$k$}
			\end{axopicture}
	\end{minipage}
	&\quad&
	\frac{i}{k^2}\delta_{ab}
	\end{align}

规范玻色子间三次相互作用:\\
\begin{align}\label{equ:app.c.18}
	\begin{minipage}[c][12ex][c]{8em}
		\begin{axopicture}(80,80)
			\Black{\Photon(0,10)(40,40){2}{7}}
			\Black{\Photon(80,10)(40,40){2}{7}}
			\Black{\Photon(40,80)(40,40){2}{7}}
			\Text(0,5){$c,\lambda$}
			\Text(80,5){$b,\nu$}
			\Text(40,85){$a,\mu$}
			%%%%%%%%%%%%%%%%%
			\Text(50,60){$k_1$}
			\Text(30,20){$k_3$}
			\Text(70,30){$k_2$}
			%%%%%%%%%%%%%%%
			\Line[arrow,arrowscale=1.2](40,62)(40,58)
			\Line[arrow,arrowscale=1.2](18,23)(22,27)
			\Line[arrow,arrowscale=1.2](62,23)(58,27)
		\end{axopicture}
	\end{minipage}
	&\quad&
	gf^{abc}[g_{\mu\nu}(k_1-k_2)_\lambda+g_{\nu\lambda}(k_2-k_3)_\mu
	+g_{\lambda\mu}(k_3-k_1)_\nu]
	\end{align}

规范玻色子间四次相互作用:\\
\begin{tabular}{ccc}
	\begin{minipage}[c][25ex][c]{10em}
		\begin{axopicture}(90,90)
			\Black{\Photon(0,0)(80,80){2}{13}}
			\Black{\Photon(80,0)(0,80){2}{13}}
			\Text(-5,-5){$d,\rho$}
			\Text(-5,85){$a,\mu$}
			\Text(85,85){$b,\nu$}
			\Text(85,-5){$c,\lambda$}
		\end{axopicture}
	\end{minipage}
	&\quad&
	\begin{minipage}[c][25ex][c]{18em}
		\begin{align}\label{equ:app.c.19}
		-ig^2[f^{abe}f^{cde}(g_{\mu\lambda}g_{\nu\rho}
		-g_{\mu\rho}g_{\nu\lambda}) \notag \\
		+f^{ade}f^{cbe}(g_{\mu\lambda} 
		g_{\nu\rho}-g_{\mu\nu}g_{\rho\lambda})  \\
		+f^{ace}f^{bde}(g_{\mu\nu}g_{\lambda\rho}
		-g_{\mu\rho}g_{\nu\lambda})] \notag 
		\end{align}
	\end{minipage}
\end{tabular}

规范玻色子与鬼场耦合:\\
\begin{align}\label{equ:app.c.20}
	\begin{minipage}[c][15ex][c]{8em}
		\begin{axopicture}(100,70)
			\Black{\Line[arrow,dash](50,30)(0,0)}
			\Black{\Line[arrow,dash](100,0)(50,30)}
			\Black{\Photon(50,30)(50,70){2}{7}}
			%%%%%%%%%%%%%%%%%
			\Text(50,75){$c,\mu$}
			\Text(25,8){$a$}
			\Text(75,8){$b$}
			\Text(15,18){$p$}
		\end{axopicture}
	\end{minipage}
	&\quad&
	gf^{abc}p^\mu
	\end{align}


\section{散射截面和衰变率}

给出过程$p_1+p_2\rightarrow k_1+k_2+\cdots+k_n$的费曼振幅
$\mathfrak{M}$,其微分散射截面为
\begin{equation}\label{equ:app.c.21}
	d\sigma=\frac{1}{|\vec{v}_1-\vec{v}_2|\mathfrak{E}(p_1)\mathfrak{E}(p_2)}
	\frac{d^3k_1}{(2\pi)^3\mathfrak{E}(k_1)}\cdots
	\frac{d^3k_n}{(2\pi)^3\mathfrak{E}(k_n)}(2\pi)^4
	\delta^{(4)}(p_1+p_2-\sum_{i=1}^nk_i)|\mathfrak{M}|^2
\end{equation}
这里$\vec{v}_1$和$\vec{v}_2$表示入射粒子的速度。
玻色子的能量因子$\mathfrak{E}(p)=2\sqrt{\vec{p}^2+m^2}$
和费米子的能量因子$\mathfrak{E}(p)=\sqrt{\vec{p}^2+m^2}/m$
来自\ref{chap:I.8}和\ref{chap:II.2}中产生湮灭算符的不同的归一化。

对质量为$M$的粒子的衰变,其在自身静止参照系下微分衰变率为
\begin{equation}\label{equ:app.c.22}
	d\Gamma=\frac{1}{2M}\frac{d^3k_1}{(2\pi)^3\mathfrak{E}(k_1)}
	\cdots\frac{d^3k_n}{(2\pi)^3\mathfrak{E}(k_n)}(2\pi)^4
	\delta^{(4)}(P-\sum_{i=1}^nk_i)|\mathfrak{M}|^2
\end{equation}

\end{document}