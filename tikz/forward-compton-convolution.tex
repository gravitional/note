\documentclass[tikz,border=10pt]{standalone}
\usepackage{tikz}
\usetikzlibrary{calc}
\usetikzlibrary{positioning}
\usetikzlibrary {arrows.meta}
\usepackage{tikz-feynman}
\begin{document}

\tikzset{
% 画三重线的代码
pics/tripleLine/.style n args={3}{
%%-----subpicture 代码
code={
\draw ($ #1+(0,#3) $) -- ($ #2 + (0,#3) $);
\draw ($ #1 - (0,#3) $) -- ($ #2 -(0,#3) $);
\draw ($ #1+(0,0) $) -- ($ #2 + (0,0) $);
\draw[-{Latex[length=3mm]}]  #1 --($ #1!.55!#2 $);
}},
}

\begin{tikzpicture}
    \begin{feynman}
        %% 上面的 blob
        \vertex (a0) at (0,0){a0};
        %% 下面的 blob
        \vertex[above right =-2.3 and 0 of a0] (b0){};
        %%
        \vertex[above right =0.2  and -0.6 of a0] (a1){};
        \vertex[above right =0.2  and +0.6 of a0] (a2){};
        \vertex[above right =1.6  and 2.5 of a0] (a3){};
        \vertex[above right =1.6  and -2.5 of a0] (a4){};
        %% 介子 blob
        \vertex[above right =-0.2  and -0.6 of a0] (a1d){a1d};
        \vertex[above right =-0.2  and +0.6 of a0] (a2d){a2d};
        % 重子blob 内部
        \vertex[above right =0.3  and -0.6 of b0] (b1u){};
        \vertex[above right =0.3  and +0.6 of b0] (b2u){};
        %%
        \vertex[above right =0 and -1.7 of b0] (ba1){};
        \vertex[above right =0 and +1.7 of b0] (ba2){};
        % 重子线左右
        \vertex[above right =0  and -3 of b0] (b1){};
        \vertex[above right =0  and +3 of b0] (b2){};
        % 对各个顶点连线
        \diagram*{
        (b1) --[opacity=1,momentum'=\(P\)] (ba1);
        (ba2)--[opacity=1,momentum'=\(P\)](b2);
        { [edge= fermion]
        (b1u) --[momentum=\(k\),edge label'=\(i\)] (a1d),
        (a2d)--[momentum=\(k\),edge label'=\(j\)](b2u),
        };
        % 光子线
        {[edge=photon]
        (a4) --[momentum=\(q\)] (a1)--(a2)--[momentum=\(q\)](a3),
        };
        };
        % 画椭圆 blob
        \filldraw[fill=blue!12,draw=black] (a0)ellipse[x radius=1.7,y radius=0.6,anchor=center];
        \filldraw[fill=blue!12,draw=black] (b0)ellipse[x radius=1.7,y radius=0.6,anchor=center];
        % 文字
        \node[above right =0  and 0 of a0] (t1){$t_{ij}^{\mu\nu}(k,q)$};
        \node[above right =0  and 0 of b0] (t2){$\chi_{ij}(k,P)$};
        %
        \pic {tripleLine = {(b1)}{(ba1)}{2pt}};
        \pic {tripleLine = {(ba2)}{(b2)}{2pt}};
    \end{feynman}
\end{tikzpicture}
\end{document}