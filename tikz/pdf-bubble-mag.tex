\documentclass[tikz,border=10pt]{standalone}
\usepackage{tikz}
\usetikzlibrary{positioning}
\usepackage{tikz-feynman}
\begin{document}

\begin{tikzpicture}
	\begin{feynman}
		%% fig j
		\vertex (j1) at (0,0);
		\vertex[right =1cm  of j1] (j2);
		\vertex[right =2cm  of j1,shape=rectangle, minimum size=0.1cm,draw,anchor=center] (j3){};
		\vertex[right =3cm  of j1] (j4);
		\vertex[right =4cm  of j1] (j5);
		\vertex[above =1.4 cm  of j3,crossed dot,anchor=center] (j6){};
		\node[above =1.9 cm  of j5] {};
		% 对各个顶点连线
		\diagram*{
		{ [edge= fermion]
				(j1) --  (j3)--(j5),
			},
		%介子连线
		{ [edge= charged scalar]
		(j3) --[half left](j6)--[half left](j3),
		}
		};
	\end{feynman}
\end{tikzpicture}


\end{document}