\documentclass[tikz,border=10pt]{standalone}
\usepackage{tikz}
\usetikzlibrary{positioning}
\usepackage{tikz-feynman}
\begin{document}

\begin{tikzpicture}
	\begin{feynman}
		%% fig h
		\vertex (i1) at (0,0);
		\vertex[right =1cm  of i1] (i2);
		\vertex[right =2cm  of i1, dot,anchor=center] (i3){};
		\vertex[right =3cm  of i1] (i4);
		\vertex[right =4cm  of i1] (i5);
		\vertex[above =-1.4cm  of i3] (i6);
		\node[above =0.5cm  of i5] {};
		% 对各个顶点连线
		\diagram*{
			{ [edge= fermion]
					(i1) --  (i3)--(i5),
				},
			% 介子连线
			% { [edge= charged scalar]
			% (h3) --[half left ](h6)--[half left](h3),
			% }
		};
		\draw  (i3) edge [anti charged scalar,loop, min distance=3.5cm] (i3);
	\end{feynman}
\end{tikzpicture}


\end{document}
