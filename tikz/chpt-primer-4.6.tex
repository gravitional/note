\documentclass[tikz,border=10pt]{standalone}
\usepackage{tikz}
\usepackage{tikz-feynman}
\begin{document}

\begin{tikzpicture}[x=2.5cm,y=2.5cm]
	\begin{feynman}
		%% fig a
		\vertex (a1) at (0,0);
		\vertex[above right = 0 and 1 of a1] (a2);
		\vertex[above right = 0 and 3 of a1] (a3);
		\vertex[above right = 0 and 4 of a1] (a4);
		\vertex[above right = 1 and 0.5 of a1] (a5);
		\vertex[above right = 1 and 3.5 of a1] (a6);
		% 对各个顶点连线
		\diagram*{
        (a1)--[fermion,momentum'=\(p\)]
        (a2)--[fermion,momentum'=\(p+q\)]
        (a3)--[fermion,momentum'=\(p^\prime\)](a4);
        (a5)--[charged scalar,momentum=\(q\)](a2);
        (a3)--[charged scalar,momentum=\(q^\prime\)](a6);
		};
	\end{feynman}
\end{tikzpicture}

\end{document}
