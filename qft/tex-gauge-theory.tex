%% LyX 2.3.5-1 created this file.  For more info, see http://www.lyx.org/.
%% Do not edit unless you really know what you are doing.
\documentclass[oneside,UTF8]{ctexbook}
\usepackage[T1]{fontenc}
\setcounter{secnumdepth}{3}
\setcounter{tocdepth}{3}
\usepackage{amsmath}
\usepackage{esint}

\makeatletter
%%%%%%%%%%%%%%%%%%%%%%%%%%%%%% User specified LaTeX commands.
% 如果没有这一句命令,XeTeX会出错,原因参见
% http://bbs.ctex.org/viewthread.php?tid=60547
% \DeclareRobustCommand\nobreakspace{\leavevmode\nobreak\ }
%%%%%%%%%%%%%%%%%+++++++++++
\usepackage{eso-pic} 
\usepackage{xcolor}
% Required for specifying colors by name 
\definecolor{ocre}{RGB}{243,102,25} 
\usepackage{enumerate} 

\usepackage{amsfonts} 
\usepackage{hep} 
\usepackage{bm} 
\usepackage{graphicx,graphics,color} 
\usepackage{simplewick} 
\usepackage{latexsym} 
\usepackage{amssymb} 
\usepackage{makeidx} 
\usepackage{amsmath} 
\usepackage{multirow} 
\usepackage{slashed} 
%%\usepackage[colorlinks,linkcolor=blue]{hyperref} 
%% \usepackage{axodraw2} 
%%++++++++++++++++++++ 
\DeclareMathOperator{\trace}{tr} 
\DeclareMathOperator{\Real}{Re} 
\DeclareMathOperator{\Imag}{Im} 
\newcommand*{\dif}{\mathop{}\!\mathrm{d}}

\makeatother

\begin{document}
\title{gauge-theory}
\author{Young}
\maketitle

\chapter{非阿贝尔规范对称性}

\section{平行移动算子}

在定域规范变换下,要使不同时空点上的场之间有物理意义的联系,{\color{red}{需要将一个时空点上的场移动到另一个点,然后进行比较}}(如加减,标量积等操作。)这时,必须引入一个引子来补偿不同时空点上,规范变换之间的差异。这种移动操作可以通过引入{\color{red}{平行移动算子}}来实现。

自由的Dirac 场(电子场)的拉氏量:$\mathcal{L}_{0}\left(x\right)=\bar{\psi}\left(x\right)\left(i\gamma^{\mu}\partial_{\mu}-m\right)\psi\left(x\right)$

整体对称性:$\psi\left(x\right)\to e^{i\alpha}\psi\left(x\right),\quad\bar{\psi}\left(x\right)\to\bar{\psi}\left(x\right)e^{-i\alpha}$

局域对称性中的$\alpha$依赖于坐标$x$。

从 x 移动到y的平行移动算子$U\left(y,x\right)$满足性质:

\[
U\left(y,x\right)\to e^{i\,\alpha\left(y\right)}U\left(y,x\right)e^{-i\,\alpha\left(x\right)}
\]

普通导数:$n^{\mu}\partial_{\mu}\psi\left(x\right)=\lim_{\varepsilon\to0}\frac{1}{\varepsilon}\left(\psi\left(x+n\epsilon\right)-\psi\left(x\right)\right)$

由于涉及到两个不同时空点上场的差。所以定域规范变换在两个时空点上是不同的,所以直接相减没有很好的定义。

协变导数:$n^{\mu}D_{\mu}\psi\left(x\right)=\lim_{\varepsilon\to0}\frac{1}{\varepsilon}\left(\psi\left(x+n\epsilon\right)-U\left(x+n\varepsilon,x\right)\psi\left(x\right)\right)$

相当于移动到一起再相减。将无穷小平行移动算子在单位元附近作展开

\[
U\left(x+n\varepsilon,x\right)=1-i\varepsilon en^{\mu}A_{\mu}\left(x\right)+O\left(\varepsilon^{2}\right)
\]

含义:变化量正比于移动距离$\varepsilon n$,比例系数为$A_{\mu}\left(x\right)$。

\[
D_{\mu}\psi\left(x\right)=\partial_{\mu}\psi\left(x\right)+ieA_{\mu}\left(x\right)\psi\left(x\right)=\left(\partial_{\mu}+ieA_{\mu}\left(x\right)\right)\psi\left(x\right)
\]

这实际上给出了电磁场的几何含义:

定域的$U\left(1\right)$对称性自然引入了$U\left(1\right)$规范场,它是一种联络场。

\subsection{有限平移算子}

如果将无穷小平行移动沿某条路径$\mathcal{C}$从$x$到$y$连续作用,则会生成一个有限的平行移动算子$U\left(y,x;\mathcal{C}\right)$:

\begin{align*}
U\left(x+\Delta x,x\right) & =1-ie\Delta x^{\mu}A_{\mu}\left(x\right)+O\left(\Delta x^{2}\right)\\
 & \approx\exp\left[-ie\int_{x}^{x+\Delta x}\dif y^{\mu}A_{\mu}\left(y\right)\right]
\end{align*}

\begin{align*}
U\left(y,x;\mathcal{C}\right) & =U\left(y,y-\Delta x\right)U\left(y-\Delta x,y-2\Delta x\right)\cdots U\left(x+\Delta x,x\right)\\
 & \equiv\mathcal{P}\exp\left[-ie\int_{x,\mathcal{C}}^{y}\dif z^{\mu}A_{\mu}\left(z\right)\right]
\end{align*}

有限的平行移动算子$U\left(y,x;\mathcal{C}\right)$是和路径$\mathcal{C}$有关的,且满足如下关系:

\begin{align*}
U\left(y,x;\mathcal{C}\right)U\left(x,y;\mathcal{C}\right) & =1\\
U\left(y,x;\mathcal{C}\right)^{-1}=U\left(x,y;\mathcal{C}\right) & =U^{\dagger}\left(y,x;\mathcal{C}\right)
\end{align*}

非定域规范不变算子(Wilson line):$\bar{\psi}\left(y\right)\Gamma U\left(y,x;\mathcal{C}\right)\psi\left(x\right)$

规范场强是规范不变的:

\begin{align*}
U\left(x,x;\mathcal{C}\right) & =U\left(x,y;\mathcal{C}_{1}\right)U\left(y,x;\mathcal{C}_{2}\right)\cdots U\left(x+\Delta x,x\right)\\
 & \equiv\mathcal{P}\exp\left[-ie\oint_{\mathcal{C}}\dif z^{\mu}A_{\mu}\left(z\right)\right]=\exp\left[-ie\iint_{\partial\mathcal{C}}\dif\sigma^{\mu\nu}F_{\mu\nu}\right]
\end{align*}

规范场强:

\[
F_{\mu\nu}\left(x\right)=\partial_{\mu}A_{\nu}\left(x\right)-\partial_{\nu}A_{\mu}\left(x\right)
\]
是规范不变的。

\section{非阿贝尔规范对称性–Yang–Mills 理论}

假设两种费米子场$\psi_{1}\left(x\right)$and $\psi_{1}\left(x\right)$,它们的质量相同,则它们的自由拉氏量可以写作
\[
\mathcal{L}_{0}=\bar{\psi}_{1}\left(i\gamma^{\mu}\partial-m\right)\psi_{1}+\bar{\psi}_{2}\left(i\gamma^{\mu}\partial-m\right)\psi_{2}\equiv\bar{\psi}\left(i\gamma^{\mu}\partial-m\right)\psi
\]
\begin{align*}
\psi\left(x\right) & =\left(\begin{array}{c}
\psi_{1}\left(x\right)\\
\psi_{2}\left(x\right)
\end{array}\right)
\end{align*}

$\mathcal{L}_{0}$在整体的$SU\left(2\right)$变换下不变:
\[
\begin{array}{ccc}
\psi\left(x\right)\to V\psi\left(x\right) & \bar{\psi}\left(x\right)\to\bar{\psi}\left(x\right)V^{\dagger}\\
V,V^{\dagger}\in SU\left(2\right), & VV^{\dagger}=V^{\dagger}V=1 & \det V=1
\end{array}
\]
这个体系就有整体的$SU\left(2\right)$对称性。

$SU\left(2\right)$群:二维复空间的幺正幺模矩阵群,三个生成元$t^{a},a=1,2,3$
\[
\left[t^{a},t^{b}\right]=i\varepsilon^{abc}t^{c},\qquad V=e^{i\theta^{a}t^{a}}\in SU\left(2\right)
\]

对于$n$维表示,生成元可以表示成$n\times n$的矩阵,我们经常用$T^{a}$来表示,以与基础表示矩阵$t^{a}$区分,相应的$SU\left(2\right)$的变换矩阵为
\[
V_{n\times n}=\exp[i\theta^{a}T^{a}]
\]


\subsection{SU(N)定域规范对称性}

如果仿照QED的情形,我们将上述整体非阿贝尔变换定域化,即令群参数$\theta^{a}$是时空的函数,

\[
V\left(x\right)\in SU\left(2\right),\qquad V\left(x\right)=\exp^{i\theta^{a}\left(x\right)t^{a}}
\]

$\mathcal{L}_{0}\left(x\right)$在这种定域变换下,不再是不变的(偏导数也作用于$V\left(x\right)$)

\begin{align*}
\mathcal{L}_{0}\left(x\right) & \to\bar{\psi}\left(x\right)V^{\dagger}\left(x\right)\left(i\gamma^{\mu}\partial_{\mu}\right)V\left(x\right)\psi\left(x\right)-m\bar{\psi}\left(x\right)\psi\left(x\right)\\
 & \bar{\psi}\left(x\right)\left(i\gamma^{\mu}\partial_{\mu}-m\right)\psi\left(x\right)+i\bar{\psi}\left(x\right)\gamma^{\mu}V^{\dagger}\left(x\right)\left(\partial_{\mu}V\left(x\right)\right)\psi\left(x\right)\\
 & \mathcal{L}_{0}\left(x\right)++i\bar{\psi}\left(x\right)\gamma^{\mu}V^{\dagger}\left(x\right)\left(\partial_{\mu}V\left(x\right)\right)\psi\left(x\right)
\end{align*}

平行移动算子,从$x$移动到$y$的平行移动算子$U\left(y,x\right)$满足性质:

\[
U\left(y,x\right)\to V\left(y\right)U\left(y,x\right)v^{\dagger}\left(x\right),V\left(x\right)=e^{i\theta^{a}\left(x\right)t^{a}}
\]

无穷小平行移动算子:在单位元附近作无穷小展开
\begin{align*}
U\left(x+n\varepsilon,x\right) & =I+i\varepsilon gn^{\mu}A_{\mu}\left(x\right)+O\left(\varepsilon\right)\\
 & =I+i\varepsilon gn^{\mu}A_{\mu}^{a}\left(x\right)t^{a}+O\left(\varepsilon^{2}\right)
\end{align*}

协变导数:
\begin{align*}
n^{\mu}D_{\mu}\psi\left(x\right) & =\lim_{\varepsilon\to0}\frac{1}{\varepsilon}\left[\psi\left(x+n\varepsilon\right)-U\left(x+n\varepsilon,x\right)\psi\left(x\right)\right]\\
 & =\lim_{\varepsilon\to0}\frac{1}{\varepsilon}\left[\psi\left(x\right)+\varepsilon n^{\mu}\partial_{\mu}\psi\left(x\right)-\psi\left(x\right)-ig\varepsilon n^{\mu}A_{\mu}^{a}\left(x\right)t^{a}\psi\left(x\right)\right]\\
 & =n^{\mu}\left(\partial_{\mu}-igA_{\mu}^{a}\left(x\right)t^{a}\right)\psi\left(x\right)
\end{align*}
\[
D_{\mu}=\partial_{\mu}-igA_{\mu}\left(x\right)=\partial_{\mu}-igA_{\mu}^{a}\left(x\right)t^{a}
\]
$g$:规范耦合常数

$A_{\mu}\left(x\right)=A_{\mu}^{a}\left(x\right)t^{a}$称为规范场,它是一种联络场。

$A_{\mu}\left(x\right)$的规范变换性质:
\begin{align*}
\left.\begin{array}{c}
U\left(y,x\right)\to V\left(y\right)U\left(y,x\right)v^{\dagger}\left(x\right)\\
U\left(x+n\varepsilon,x\right)=I+i\varepsilon gn^{\mu}A_{\mu}\left(x\right)+O\left(\varepsilon^{2}\right)
\end{array}\right\}  & \to\\
I+ig\varepsilon n^{\mu}A_{\mu}\left(x\right)\to & I+ig\varepsilon n^{\mu}\left(V\left(x\right)A_{\mu}\left(x\right)V^{\dagger}\left(x\right)+\frac{i}{g}V\left(x\right)\partial_{\mu}V^{\mu}\left(x\right)\right)
\end{align*}
$A_{\mu}\left(x\right)$的规范变换性质(重要):
\[
A_{\mu}\left(x\right)\to V\left(x\right)A_{\mu}\left(x\right)V^{\dagger}\left(x\right)+\frac{i}{g}V\left(x\right)\partial_{\mu}V^{\dagger}\left(x\right)
\]
即$A_{\mu}\left(x\right)$不是规范协变的,而$D_{\mu}\psi\left(x\right)$是规范协变的
\[
D_{\mu}\left(x\right)\psi\left(x\right)\to V\left(x\right)D_{\mu}\left(x\right)\psi\left(x\right)
\]

进一步地有
\begin{align*}
D_{\mu}\psi\left(x\right) & \to V\left(x\right)D_{\mu}\left(x\right)\psi\left(x\right)=V\left(x\right)D_{\mu}\left(x\right)\left(v^{\dagger}\left(x\right)V\left(x\right)\right)\psi\left(x\right)\\
 & =V(x)D_{\mu}(x)V^{\dagger}(x)V(x)\psi(x)
\end{align*}
\[
D_{\mu}(x)\to V(x)D_{\mu}(x)V^{\dagger}(x)
\]

end of file

end of file
\begin{thebibliography}{1}
\bibitem{ref61}ref61
\end{thebibliography}

\end{document}
